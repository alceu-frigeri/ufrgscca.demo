%%%==============================================================================
%%
%% Auxiliary (sample) local.tex file
%%
%%%==============================================================================
%

%%
\ActivitySetCoord[course]{Alceu Heinke Frigeri}[m]
% o nome do curso pode ser redefinido (ex. para Monografias)
%\ActivitySet{CCA}{Engenharia de Controle e Automação}

\ActivitySetCoord[internship]{Rafael Comparsi Laranja}[m]


%%
%%Outro elementos podem ser ajustados, se necessário
%%
%\SetHeadings{%
%    university              = {Universidade Federal do Rio Grande do Sul} ,
%    acronym                 = {UFRGS} ,
%    unit                    = {Escola de Engenharia} ,
%    course                  = {Engenharia de Controle e Automação} ,
%    course.title            = {Bacharel em Engenharia de Controle e Automação} ,
%    departament             = {Depto. de Sistemas Elétricos de Automação e Energia} ,
%}


% no caso de um relatório de disciplina
%\class{ENG10004}{Sistemas de Controle I}


% o local de realização do trabalho pode ser especificado (ex. para Monografias)
% com o comando \location:
%\location{São José dos Campos}{SP}


%%%%%%%%%%%%%%%%%%%%%%%%%%%%%%%%%%%%%%
%%%%%%%%%%%%%%%%%%%%%%%%%%%%%%%%%%%%%%
%%%%%%%%%%%%%%%%%%%%%%%%%%%%%%%%%%%%%%


% Informações gerais
%

\student{Formanda}{Nome da Aluna}[f]
\studentinfo{No. cartão1}{email}

% alguns documentos podem ter varios autores:
%\student{Formando}{Nome do Aluno}[m]
%\studentinfo{No. cartão2}{email}


\tutor[Prof.~Dr.]{do Tutor}{Nome}[m]
\tutorinfo{UFRGS}{Instituição -- Cidade, País}{email}{ramal}

\supervisor[Eng. Mec/Eletr/Ctrl]{do Supervisor}{Nome}[m]
\supervisorinfo{crea}{posição/cargo}{email}{ramal}

\internship{Empresa Ltda.}{P\&D}{10/10/22}{20/12/22}{2 Meses}


% a data deve ser a da entrega do relatório; se nao especificada, são usados
% mes e ano correntes
%\pubdate{Fevereiro}{2004}


