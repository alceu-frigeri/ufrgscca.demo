%%%==============================================================================
%%
%% Auxiliary (sample) local.tex file
%%
%%%==============================================================================
%




\professor[alceu]{Heinke Frigeri}{Alceu}{alceu.frigeri@ufrgs.br}{4290}[m]
\professor[aurelio]{Tergolina Salton}{Aurélio}{aurelio.salton@ufrgs.br}{4281}[m]
\professor[denardi]{Denardi Huff}{Daniel}{danielhuff987@gmail.com}{phone}[m]
\professor[edison]{Pignaton de Freitas}{Edison}{edisonpf@gmail.com}{7033}[m]
\professor[fabiano]{Disconzi Wildner}{Fabiano}{wildner@mecanica.ufrgs.br}{3773}[m]
\professor[farenzena]{Farenzena}{Marcelo}{farenz@enq.ufrgs.br}{3918}[m]
\professor[feldman]{Feldman}{Max}{email}{phone}[m]
\professor[goetz]{Götz}{Marcelo}{mgoetz@ufrgs.br}{4448}[m]
\professor[ivan]{Müller}{Ivan}{ivan.muller@ufrgs.br}{4484}[m]
\professor[jeferson]{Vieira Flores}{Jeferson}{jeferson.flores@ufrgs.br}{3291}[m]
\professor[jorge]{Trierweiler}{Jorge Otávio}{jorge.o.trierweiler@gmail.com}{4072}[m]
\professor[heraldo]{Amorim}{Heraldo José de}{amorim@ufrgs.br}{3361}[m]
\professor[jm]{Gomes da Silva Jr}{João Manoel}{jmgomes@ufrgs.br}{3140}[m]
\professor[joaonetto]{Netto}{João Cesar}{netto@inf.ufrgs.br}{11500504}[m]
\professor[laranja]{Comparsi Laranja}{Rafael Antônio}{rafael.laranja@ufrgs.br}{3749}[m]  %00028864@ufrgs.br
\professor[luciola]{Campestrini}{Lucíola}{luciola@ufrgs.br}{4472}[f]
\professor[mario]{Sobczyk Sobrinho}{Mário Roland}{mario.sobczyk@ufrgs.br}{4541}[m]
\professor[nardelli]{Camargo Nardelli}{Vítor}{email}{phone}[m]
\professor[pedro]{Bolognese Fernandes}{Pedro Rafael}{pedro@enq.ufrgs.br}{4074}[m]
\professor[perondi]{Perondi}{Eduardo André}{eduardo.perondi@ufrgs.br}{3567}[m]
\professor[renato]{Bayan Henriques}{Renato Ventura}{renatobayan@gmail.com}{phone}[m]
\professor[valner]{Brusamarello}{Valner João}{brusamarello.valner@gmail.com}{3515}[m]
\professor[walter]{Fetter Lages}{Walter}{fetter@ece.ufrgs.br}{4287}[m]
\professor[yachel]{Mileski}{Yachel}{yachel.mileski@gmail.com}{phone}[m]





%%%%%%%%%
%%
%% TCC - I
%%
%%%%%%%%%

\ActivitySelect{tccI}

%%%%%%%%
\ActivitySetNewEvent{Apresentacao}
    {Apresentação da disciplina. Informações gerais. Reunião presencial com início às 18:30h, na sala 301 da eletro.\newline {\bf Presença obrigatória.}}

\ActivitySetNewEvent{Matricula}
    {Requerimento de matrícula. O(A) aluno(a) deve abrir processo/petição de matrícula em TCC no porta TuaUFRGS, adicionando ao mesmo o \emph{Formulário de Requerimento de Matrícula}, já com o resumo do trabalho e, se necessário, \emph{Requerimento de Autorização de Co-Orientação}, assinados pelo orientador.
    \newline \textbf{IMPORTANTE:} Na Petição, TuaUFRGS, informar no campo curso: p/ \emph{Coordenadoria da Atividade de TCC do Curso em Eng. Controle e Automação} - \textbf{CATCC-CCA}
     \newline{\bf Obs.~1}}

\ActivitySetNewEvent{EncontroI}
    {Encontro via Teams: Como redigir um TCC, melhores práticas. Início às 18:30h.}% \newline {\bf Presença obrigatória.}}

\ActivitySetNewEvent{IndicaBanca}
    {Entrega, VIA MOODLE, do  \emph{Formulário de Indicação da Banca Examinadora} e, se necessário, \emph{Requerimento de Autorização de Co-Orientação}, assinados pelo orientador. \newline{\bf Obs.~2}}

\ActivitySetNewEvent{EntregaTCC}
    {Entrega, VIA MOODLE, do Trabalho, juntamente com os \emph{Formulários de Avaliação do Trabalho} (um por membro da banca) e \emph{Formulário de Aprovação para Avaliação}, assinado pelo Orientador.}

\ActivitySetNewEvent{Bancas}
    {Análise dos Trabalhos pelas Bancas.}% \newline {\bf Presença obrigatória.}}

\ActivitySetNewEvent{Correcoes}
    {Entrega, VIA MOODLE, da versão final corrigida do trabalho e o \emph{Formulário de Aprovação das Correções}.\newline{\bf Obs.~3}}

\ActivitySetNewEvent{Exame}
    {Para os TCCs em RECUPERAÇÃO, entrega, VIA MOODLE, da versão final corrigida do trabalho e do \emph{Formulário de Aprovação das Correções}.\newline{\bf Obs.~4}}


\SetTerms{%
  notes.calendarI            = {%
    \begin{description}
      \item[Obs. 1]  Caso o(a) aluno(a) já tenha um processo aberto de TCC, reusar o mesmo. Neste caso, encaminhar o novo requerimento de matrícula \ActivityCoord{tccII}{carticle} coordenador\ActivityCoord{tccII}{narticle} da atividade (email: \textbf{tcc-cca@ufrgs.br}), indicando o número do processo original.
      \item[Obs. 2]  O formulário deve conter as assinaturas de concordância, ou email de aceites, em participar da banca da parte dos professores convidados.
      \item[Obs. 3]  O \emph{Formulário de Aprovação das Correções} deve estar assinado pelo(a) orientador(a). Para trabalhos em que o campo “Revisarei o trabalho depois de corrigido” tiver sido assinalado por algum membro da banca, a assinatura do(a) mesmo(a) também deverá constar no \emph{Formulário de Aprovação das Correções}.
      \item[Obs. 4]  Trabalho em \textbf{RECUPERAÇÃO} é aquele cuja nota final da banca é insuficiente para aprovação OU aquele em que o campo “\textbf{Necessita ser revisto/re-escrito}” do formulário de correções foi marcado por algum membro da banca. Neste caso, o \emph{Formulário de Aprovação das Correções} deve estar assinado tanto pelo(a) orientador(a) como pelo(s) membro(s) da banca designado(s) para reavaliar o trabalho, incluindo obrigatoriamente aqueles que tiverem marcado o campo “Revisarei o trabalho depois de corrigido”.\\\hrule
      \item[IMPORTANTE]\  O(A) aluno(a) estará \textbf{REPROVADO(A)} \emph{se falhar na entrega de qualquer um dos documentos aqui citados dentro dos prazos estipulados}.
    \end{description}
  } ,
}


%%%%%%%%%
%%
%% TCC - II
%%
%%%%%%%%%

\ActivitySelect{tccII}

%%%%%%%%%
\ActivitySetNewEvent{Apresentacao}
    {Apresentação da disciplina. Informações gerais. Reunião presencial com início às 18:30h, na sala 301 da eletro.\newline {\bf Presença obrigatória.}}

\ActivitySetNewEvent{Matricula}
    {Requerimento de matrícula. O(A) aluno(a) deve encaminhar o formulário de Requerimento de matrícula e, se necessário, \emph{Requerimento de Autorização de Co-Orientação}, assinados pelo orientador,  \ActivityCoord{tccII}{carticle} coordenador\ActivityCoord{tccII}{narticle} da atividade (email: \textbf{tcc-cca@ufrgs.br}), indicando o número do processo original (TCC-I).}% \newline{\bf Obs.~1}}

\ActivitySetNewEvent{EncontroI}
    {Encontro via Teams: Como redigir um TCC, melhores práticas. Início às 18:30h.}% \newline {\bf Presença obrigatória.}}

\ActivitySetNewEvent{IndicaBanca}
    {Entrega, VIA MOODLE, do \emph{Formulário de Indicação da Banca Examinadora} assinado pelo orientador. \newline{\bf Obs.~1}}

\ActivitySetNewEvent{Seminarios}
    {
     Seminários de Acompanhamento. A marcar.
    }

\ActivitySetNewEvent{EntregaTCC}
    {Entrega de uma cópia do Trabalho de Conclusão a cada membro da banca, juntamente com os \emph{Formulários de Avaliação} e \emph{Formulários de correções} do trabalho. Entrega, VIA MOODLE, de uma cópia eletrônica do trabalho juntamente com o \emph{Formulário de Aprovação para Apresentação} assinado pelo orientador.}

%\ActivitySetNewEvent{eventF}
%    {Encontro via Teams: Como apresentar um TCC, melhores técnicas. Início às 18:30h.}% \newline {\bf Presença obrigatória.}}

\ActivitySetNewEvent{Bancas}
    {Apresentações dos Trabalhos de Conclusão de Curso. \newline{\bf Obs.~2}}

\ActivitySetNewEvent{FormCorrecoes}
    {Entrega, VIA MOODLE, dos \emph{Formulários de correções} dos membros da banca.}

\ActivitySetNewEvent{Correcoes}
    {Entrega, VIA MOODLE, da versão final corrigida do trabalho e o \emph{Formulário de Aprovação das Correções}. Entrega do formulário de autorização para publicação (biblioteca) assinado, VIA MOODLE. \newline{\bf Obs.~3}}

\ActivitySetNewEvent{Exame}
    {Para os TCCs em RECUPERAÇÃO, entrega, VIA MOODLE, da versão final corrigida do trabalho e o \emph{Formulário de Aprovação das Correções}. Entrega do formulário de autorização para publicação (biblioteca) assinado, VIA MOODLE. \newline{\bf Obs.~4}}


\SetTerms{%
  notes.calendarII            = {%
    \begin{description}
      \item[Obs. 1]  O formulário deve conter as assinaturas de concordância, ou email de aceite, em participar da banca da parte dos professores convidados.
      \item[Obs. 2]  As bancas serão marcadas após a entrega do \emph{Formulário de Aprovação para Apresentação}, de acordo com a disponibilidade dos professores convidados.
      \item[Obs. 3]  O \emph{Formulário de Aprovação das Correções} deve estar assinado pelo(a) orientador(a). Para trabalhos em que o campo “Revisarei o trabalho depois de corrigido tiver sido assinalado por algum membro da banca, a assinatura do(a) mesmo(a) também deverá constar no \emph{Formulário de Aprovação das Correções}.
      \item[Obs. 4]  Trabalho em \textbf{RECUPERAÇÃO} é aquele cuja nota final quando da defesa é insuficiente para aprovação OU aquele em que o campo “\textbf{Necessita ser revisto/re-escrito}” do formulário de correções foi marcado por algum membro da banca. Neste caso, o \emph{Formulário de Aprovação das Correções} deve estar assinado tanto pelo(a) orientador(a) como pelo(s) membro(s) da banca designado(s) para reavaliar o trabalho, incluindo obrigatoriamente aqueles que tiverem marcado o campo “Revisarei o trabalho depois de corrigido”.\\\hrule
      \item[IMPORTANTE]\  O(A) aluno(a) estará \textbf{REPROVADO(A)} \emph{se falhar na entrega de qualquer um dos documentos aqui citados dentro dos prazos estipulados}.
    \end{description}
  } ,
}





%%%%%%%%
%%
%% check list for TCC-II
%%
%%%%%%%%
\ActivitySelect{tccII}


\checkdef{L1C1}{semiOK}{Apres. Seminário OK}
\checkdef{L2C1}{semiNao}{Apres. Seminário :/}
%\checkdef{L3C1}{}{}
%\checkdef{L4C1}{}{}
%\checkdef{L5C1}{}{}

\checkdef{L1C2}{board}{Banca def.}
\checkdef{L2C2}{board-date}{Data defesa}
%\checkdef{L3C2}{}{}
%\checkdef{L3C2}{}{}
%\checkdef{L5C2}{}{}

%\checkdef{L1C3}{tcc-board}{TCC banca}
\checkdef{L2C3}{advisorsOK}{Aprov. p/Apres.}
%\checkdef{L3C3}{receipts}{Recibos Banca}
%\checkdef{L4C3}{}{}
%\checkdef{L5C3}{}{}

\checkdef{L1C4}{rectifyI}{Form. Correções I}
\checkdef{L2C4}{rectifyII}{Form. Correções II}
\checkdef{L3C4}{rectifyIII}{Form. Correções III}
%\checkdef{L4C4}{}{}
%\checkdef{L5C4}{}{}

\checkdef{L1C5}{tcc-final}{TCC final}
\checkdef{L2C5}{approval}{Aprovação Correções}
%\checkdef{L3C5}{}{}
\checkdef{L4C5}{exam}{Em Exame}
%\checkdef{L5C5}{}{}



%%%%%%%%
%%
%% check list for TCC-I
%%
%%%%%%%%
\ActivitySelect{tccI}


\checkdef{L1C1}{banca}{Banca Aval.}
%\checkdef{L2C1}{}{}
%\checkdef{L3C1}{}{}
%\checkdef{L4C1}{}{}
%\checkdef{L5C1}{}{}

\checkdef{L1C2}{OKorientador}{Aprov. p/Aval.}
\checkdef{L2C2}{tcc}{TCC p/ Banca}
%\checkdef{L3C2}{}{}
%\checkdef{L3C2}{}{}
%\checkdef{L5C2}{}{}

%\checkdef{L1C3}{}{}
%\checkdef{L2C3}{}{}
%\checkdef{L3C3}{}{}
%\checkdef{L4C3}{}{}
%\checkdef{L5C3}{}{}

\checkdef{L1C4}{rectifyI}{Form. Correções I}
\checkdef{L2C4}{rectifyII}{Form. Correções II}
%\checkdef{L3C4}{rectifyIII}{I Form. Correções III}
%\checkdef{L4C4}{}{}
%\checkdef{L5C4}{}{}

\checkdef{L1C5}{tcc-final}{TCC final}
\checkdef{L2C5}{OKfinal}{Aprovação Correções}
%\checkdef{L3C5}{}{}
\checkdef{L4C5}{exam}{Em Exame}
%\checkdef{L5C5}{}{}



